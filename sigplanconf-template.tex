%-----------------------------------------------------------------------------
%
%               Template for sigplanconf LaTeX Class
%
% Name:         sigplanconf-template.tex
%
% Purpose:      A template for sigplanconf.cls, which is a LaTeX 2e class
%               file for SIGPLAN conference proceedings.
%
% Guide:        Refer to "Author's Guide to the ACM SIGPLAN Class,"
%               sigplanconf-guide.pdf
%
% Author:       Paul C. Anagnostopoulos
%               Windfall Software
%               978 371-2316
%               paul@windfall.com
%
% Created:      15 February 2005
%
%-----------------------------------------------------------------------------


\documentclass[preprint]{sigplanconf}
% The following \documentclass options may be useful:
% preprint       Remove this option only once the paper is in final form.
%  9pt           Set paper in  9-point type (instead of default 10-point)
% 11pt           Set paper in 11-point type (instead of default 10-point).
% numbers        Produce numeric citations with natbib (instead of default author/year).
% authorversion  Prepare an author version, with appropriate copyright-space text.
\usepackage{amsmath}
\usepackage{listings}
\usepackage{xcolor}

\colorlet{light-gray}{gray!20}
\newcommand{\cL}{{\cal L}}

\begin{document}

\special{papersize=8.5in,11in}
\setlength{\pdfpageheight}{\paperheight}
\setlength{\pdfpagewidth}{\paperwidth}

\conferenceinfo{CONF'yy}{Month d--d, 20yy, City, ST, Country}
\copyrightyear{20yy}
\copyrightdata{978-1-nnnn-nnnn-n/yy/mm}\reprintprice{\$15.00}
\copyrightdoi{nnnnnnn.nnnnnnn}

% For compatibility with auto-generated ACM eRights management
% instructions, the following alternate commands are also supported.
%\CopyrightYear{2016}
%\conferenceinfo{CONF'yy,}{Month d--d, 20yy, City, ST, Country}
%\isbn{978-1-nnnn-nnnn-n/yy/mm}\acmPrice{\$15.00}
%\doi{http://dx.doi.org/10.1145/nnnnnnn.nnnnnnn}

% Uncomment the publication rights used.
%\setcopyright{acmcopyright}
\setcopyright{acmlicensed}  % default
%\setcopyright{rightsretained}

\titlebanner{banner above paper title}        % These are ignored unless
\preprintfooter{short description of paper}   % 'preprint' option specified.

\title{Trinity: A Language for Multi-View Architecture Description and Control}%\titlenote{with optional title note}}
\subtitle{Subtitle Text, if any\titlenote{with optional subtitle note}}

\authorinfo{Name1}%\thanks{with optional author note}}
           {Affiliation1}
           {Email1}
\authorinfo{Name2 \and Name3}%\thanks{with optional author note}}
           {Affiliation2/3}
           {Email2/3}

\maketitle

\begin{abstract}
This is the text of the abstract.
\end{abstract}

% 2012 ACM Computing Classification System (CSS) concepts
% Generate at 'http://dl.acm.org/ccs/ccs.cfm'.
\begin{CCSXML}
<ccs2012>
<concept>
<concept_id>10011007.10011006.10011008</concept_id>
<concept_desc>Software and its engineering~General programming languages</concept_desc>
<concept_significance>500</concept_significance>
</concept>
<concept>
<concept_id>10003752.10010124.10010138.10010143</concept_id>
<concept_desc>Theory of computation~Program analysis</concept_desc>
<concept_significance>300</concept_significance>
</concept>
</ccs2012>
\end{CCSXML}

%%% *** COMMENTED OUT ***
%\ccsdesc[500]{Software and its engineering~General programming languages}
%\ccsdesc[300]{Theory of computation~Program analysis}
% end generated code

% Legacy 1998 ACM Computing Classification System categories are also
% supported, but not recommended.
%\category{CR-number}{subcategory}{third-level}[fourth-level]
%\category{D.3.0}{Programming Languages}{General}
%\category{F.3.2}{Logics and Meanings of Programs}{Semantics of Programming Languages}[Program analysis]

%%% COMMENTED OUT
%\keywords
%keyword1, keyword2

\section{Introduction}

\section{Design}

\begin{figure}
	
	\begin{lstlisting}[frame=single, backgroundcolor=\color{light-gray}, basicstyle=\footnotesize\ttfamily, numbers=left, numberstyle=\tiny\color{black},caption= {Simple 3-tier web application architecture}]
component Client
	port getInfo: requires CSIface

component Server
	port sendInfo: provides CSIface

external component DB
	port dbIface: target DBModule

connector JDBCCtr
	val connectionString: String

architecture 
	components
		RequestHandler ch
		DB db

connectors 
	JDBCCtr jdbcCtr

attachments
	connect rh.dbIface and db.dbIface 
		with jdbcCtr

bindings 
	sendInfo is rh.sendInfo

architecture 	
	components
		Client client
		Server server

	connectors
		JSONCtr jsonCtr

	attachments 
		Connect client.getInfo 
		and server.sendInfo with jsonCtr

	entryPoints
		Client: start
	\end{lstlisting}

\end{figure}
	
	Trinity is designed to unify software architecture design and implementation for not just a single architecture view, but in all three (module, component-and-connector, and deployment). 

	To demonstrate how Trinity makes software architecture “live” in Wyvern systems, we have implemented a simple 3-tier web application whose abridged code is shown in Figure 1. In overview, a database is accessed by a server that handles requests from a client. The example architecture contains two components, the client and server. As in more theoretical software architecture, components are runtime entities that may have ports  that act as access points to interact with other components. Our example server and client each have complementary ports, sendInfo and getInfo respectively, that enable interact; here, the server can send information to the client using a JSON connector responsible for serialization and deserialization. Note that Trinity connectors enable the joining two compatible component ports, analogous to ports in software architecture. The architecture’s attachments section actually/physically connects the client and the server using their matching ports and the JSON connector. 

	The client and server have their own component-specific architectures given before the general architecture code. Both components have corresponding ports that depend on a client-server interface, denoted by the “requires/provides CSIface” types of each port. This interface is required of a server by the client, as stated by the “requires CSIFace” type, and the server fulfills it, as shown by the “provides CSIface” server port type. The database is an external component of the server, which differs from a component in that the programmer does not provide its source code. “port dbIface: requires DBModule” The database and server are connected by a JDBC connector. This is used within a sub-architecture of the server that connects a specified request handler and the database. Finally, the bindings section of the server’s sub-architecture specifies that the client-facing port, sendInfo, of the server component is indeed the same sendInfo port of the request handler. 

	
\appendix
\section{Appendix Title}

This is the text of the appendix, if you need one.

\acks

Acknowledgments, if needed.

% The 'abbrvnat' bibliography style is recommended.

\bibliographystyle{abbrvnat}

% The bibliography should be embedded for final submission.

\begin{thebibliography}{}
\softraggedright

\bibitem[Smith et~al.(2009)Smith, Jones]{smith02}
P. Q. Smith, and X. Y. Jones. ...reference text...

\end{thebibliography}


\end{document}
